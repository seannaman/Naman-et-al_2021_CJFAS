\documentclass[]{article}
\usepackage{lmodern}
\usepackage{amssymb,amsmath}
\usepackage{ifxetex,ifluatex}
\usepackage{fixltx2e} % provides \textsubscript
\ifnum 0\ifxetex 1\fi\ifluatex 1\fi=0 % if pdftex
  \usepackage[T1]{fontenc}
  \usepackage[utf8]{inputenc}
\else % if luatex or xelatex
  \ifxetex
    \usepackage{mathspec}
  \else
    \usepackage{fontspec}
  \fi
  \defaultfontfeatures{Ligatures=TeX,Scale=MatchLowercase}
\fi
% use upquote if available, for straight quotes in verbatim environments
\IfFileExists{upquote.sty}{\usepackage{upquote}}{}
% use microtype if available
\IfFileExists{microtype.sty}{%
\usepackage[]{microtype}
\UseMicrotypeSet[protrusion]{basicmath} % disable protrusion for tt fonts
}{}
\PassOptionsToPackage{hyphens}{url} % url is loaded by hyperref
\usepackage[unicode=true]{hyperref}
\hypersetup{
            pdftitle={Diel patterns of foraging and microhabitat use by sympatric rainbow trout and bull trout: implications for adaptive differentiation and instream flow assessment},
            pdfborder={0 0 0},
            breaklinks=true}
\urlstyle{same}  % don't use monospace font for urls
\usepackage[margin=1in]{geometry}
\usepackage{graphicx,grffile}
\makeatletter
\def\maxwidth{\ifdim\Gin@nat@width>\linewidth\linewidth\else\Gin@nat@width\fi}
\def\maxheight{\ifdim\Gin@nat@height>\textheight\textheight\else\Gin@nat@height\fi}
\makeatother
% Scale images if necessary, so that they will not overflow the page
% margins by default, and it is still possible to overwrite the defaults
% using explicit options in \includegraphics[width, height, ...]{}
\setkeys{Gin}{width=\maxwidth,height=\maxheight,keepaspectratio}
\IfFileExists{parskip.sty}{%
\usepackage{parskip}
}{% else
\setlength{\parindent}{0pt}
\setlength{\parskip}{6pt plus 2pt minus 1pt}
}
\setlength{\emergencystretch}{3em}  % prevent overfull lines
\providecommand{\tightlist}{%
  \setlength{\itemsep}{0pt}\setlength{\parskip}{0pt}}
\setcounter{secnumdepth}{0}
% Redefines (sub)paragraphs to behave more like sections
\ifx\paragraph\undefined\else
\let\oldparagraph\paragraph
\renewcommand{\paragraph}[1]{\oldparagraph{#1}\mbox{}}
\fi
\ifx\subparagraph\undefined\else
\let\oldsubparagraph\subparagraph
\renewcommand{\subparagraph}[1]{\oldsubparagraph{#1}\mbox{}}
\fi

% set default figure placement to htbp
\makeatletter
\def\fps@figure{htbp}
\makeatother

\usepackage{lineno}
\linenumbers
\usepackage{setspace}\doublespacing
\setlength{\parskip}{1em}

\title{Diel patterns of foraging and microhabitat use by sympatric rainbow
trout and bull trout: implications for adaptive differentiation and
instream flow assessment}
\author{}
\date{\vspace{-2.5em}}

\begin{document}
\maketitle

\begin{center}
Sean M. Naman$^{1,2*}$, Jordan S. Rosenfeld$^3$, Alecia S. Lannan$^{1, 4}$
\end{center}

\vspace{2cm}

\begin{enumerate}
\def\labelenumi{\arabic{enumi}.}
\tightlist
\item
  Department of Geography, University of British Columbia, Vancouver,
  BC, Canada
\item
  Current affiliation: Earth to Ocean Group, Simon Fraser University,
  Burnaby, BC, Canada *Corresponding author:
  \href{mailto:sean_naman@sfu.ca}{\nolinkurl{sean\_naman@sfu.ca}},
  604-619-4274
\item
  Conservation Science Section, BC Ministry of Environment, Vancouver,
  BC Canada
\item
  Current affiliation: Ecofish Research Ltd. Vancouver, BC Canada
  \vspace{4cm}
\end{enumerate}

\raggedright\large
Key words: Foraging, predation risk, habitat suitability \newpage

\section{Abstract}\label{abstract}

Salmonids make flexible and adaptive trade-offs between foraging
efficiency and predation risk that result in variable patterns of diel
activity and habitat use. However, it remains unclear: (1) how patterns
differ among salmonid species; and (2) how this affects the
interpretation of habitat suitability models that inform instream flow
management. We combined snorkel observations with experimental additions
of cover to investigate how predation risk, cover, and bioenergetics
affect diel activity and habitat use patterns by sympatric rainbow trout
and bull trout in the Skagit River, BC, Canada. Both species foraged
primarily at dusk, supporting the well-described trade-off between
foraging efficiency and predation risk. However, only rainbow trout
responded to cover additions, suggesting that risk tolerance and the
nature of foraging-predation risk trade-offs differ between species.
Diel shifts in activity and habitat use also substantially altered
predictions of habitat suitability models, with potentially large
consequences for flow management.

\section{Introduction}\label{introduction}

Stream salmonids are well known for flexible diel activity patterns,
with feeding chronology ranging from being diurnal (day active),
nocturnal (night active), or crepuscular (active at dusk and dawn;
Alanärä and Brännäs 1997; Bradford and Higgins 2001). Microhabitat
selection is closely related to activity and also exhibits strong diel
variability, since alternative activities (e.g., feeding, resting, and
avoiding predators) require different habitat features. For example,
intermediate water velocities optimize energy intake when feeding on
drifting invertebrates, while low velocity microhabitats minimize energy
expenditure when resting (Fausch and White 1981). Similarly, structural
cover may be critical for fish to avoid visual predators during the day
(Wilzbach 1985; Harvey and White 2016), but less important during
crepuscular periods or at night when predation risk is greatly reduced.

Optimizing energy acquisition vs.~avoiding predation are the competing
endpoints of a continuum that determines realized habitat selection.
Diel patterns of activity and habitat use therefore reflect a complex
state-dependent trade-off between growth and survival, which tracks
fluctuations in foraging efficiency, predation risk, and prey abundance
(Metcalfe et al. 1999; Railsback et al. 2005). Generally, foraging in
daylight is more efficient but incurs higher predation risk (Wilzbach et
al. 1986; Fraser and Metcalfe 1997). Food abundance also has a strong
temporal signature, often increasing as light levels decline due to
elevated invertebrate drift (Bishop 1969; Naman et al. 2016).
Collectively, these fluctuating conditions, along with attributes of
individual fish (e.g., body size, energetic status), interactively
determine diel patterns of activity and habitat use. This is well
described by theory (Railsback et al. 2020a), yet the drivers of
realized patterns in natural systems, e.g., predation risk vs.~food
abundance, are challenging to disentangle. Further, different salmonid
species often exhibit contrasting activity and habitat use patterns
(Hearn 1987), which may relate to poorly understood differences in
behaviour, morphology, or physiology that determine the strength and
shape of growth-survival trade-offs.

Better understanding the underlying drivers of diel activity and habitat
use patterns across species informs both our understanding of adaptive
differentiation and management needs. First, divergent diel activity
patterns may be a critical yet underappreciated dimension of resource
partitioning that facilitates coexistence of sympatric salmonid species
(Kronfeld-Schor and Dayan 2003; Young 2004). Second, diel variation in
microhabitat use can influence the development and application of
habitat suitability models that are integral to instream flow management
and habitat restoration (Railsback et al. 2020b). These models are
commonly derived from snapshot observations of diurnal habitat use and
rarely consider temporally varying habitat requirements (but see Harris
et al. 1992; Roussel et al. 1999; Al-Chokhachy and Budy 2007). Recent
simulation modelling by Railsback et al. (2020b) indicates that ignoring
diel activity variation can introduce significant bias into habitat
suitability model predictions, and ultimately undermine management. Yet,
this bias is not well documented empirically.

Here we consider the ecological and applied implications of diel
variation in activity and habitat use by sympatric juvenile rainbow
trout \emph{Onchorhynchus mykiss} and bull trout \emph{Salvelinus
confluentes} in the Skagit River, Southwest British Columbia, Canada.
These species often exhibit contrasting patterns of diel activity, with
bull trout showing a more consistent nocturnal bias relative to rainbow
trout, which are more variable (Elliott 1973; Angradi and Griffith 1990;
Jakober et al. 2000). First, we compared diel patterns of microhabitat
use and behaviour between the two species to determine how they overlap
in space and time. Second, we combined measurements of prey abundance
with experimental additions of cover from predators to differentiate the
roles of hydraulic constraints (i.e.~velocity and depth) vs.~predation
risk in determining habitat use. We reasoned that if fish foraged near
added cover during the day, it would suggest predation risk was a key
driver of habitat use and activity patterns. As a corollary, contrasting
responses to cover between species would then indicate differences in
growth-survival trade-offs that underlie activity patterns. Third, we
compared simple habitat suitability models developed from observations
during the day, dusk, and night to assess how diel variability in
habitat use modifies habitat suitability predictions for both species.

Based on previous work on the ecology of these species, we made several
\emph{a priori} predictions. First, we expected that bull trout would be
more nocturnal than rainbow trout, i.e., that more bull trout would be
observed foraging at night (Baxter and McPhail 1997). Second, we
predicted that if predation risk were a strong influence on diurnal
habitat use and activity, experimental additions of cover that provide
shelter from predators would be colonized by daytime foraging fish. We
anticipated that this effect would be more pronounced for rainbow trout
given their greater tendency for diurnal foraging. Third, we predicted
that habitat suitability model predictions would be sensitive to time of
day, with daylight habitat use biased towards lower velocity resting
microhabitats or structural cover from predators.

\section{Methods}\label{methods}

\emph{Study Location}

We conducted our study in the upper Skagit River watershed in Southwest,
British Columbia, Canada near the confluence of the Skagit and Sumallo
Rivers. The Skagit flows south from British Columbia into Ross Lake
reservoir in the United States, eventually draining into Puget Sound.
The 99,768 ha watershed is comprised primarily of second growth
coniferous forest, with 70\% of its area protected as a Provincial Park.
We selected five 100 m reaches based on representative geomorphic
characteristics of the upper watershed and accessibility, which was
limited due to wildfires. Two reaches on the Sumallo River had lower
gradient pool-riffle morphology and predominantly well sorted gravel and
cobble substrate; the three mainstem Skagit reaches were higher
gradient, with more heterogeneous substrate composition that included
larger boulders. Mean daily temperatures ranged from 9-11°C across all
sites throughout the duration of the study (August and September 2018;
US National Park Service \emph{unpublished data}).

Rainbow trout, bull trout, and dolly varden (\emph{S. malma}) are
present in the watershed; however, dolly varden occur mainly in smaller
tributaries and were not directly observed in this study. There is
limited information about rainbow trout and bull trout populations in
the upper Skagit, but there are likely resident and adfluvial life
histories present in both species (Skagit Environmental Endowment
Commission personal communication).

\emph{Field Methods}

Our approach combined measurements of physical habitat attributes and
invertebrate prey abundance, snorkel observations of habitat use and
activity by juvenile rainbow trout and bull trout, and a manipulative
experiment of cover designed to test the influence of predation risk on
diel habitat use and behaviour. We describe each of these components in
detail below.

\emph{Habitat and Behaviour Observations} - To determine habitat
availability, we conducted systematic point transects of depth,
velocity, substrate, and cover in each reach (Appendix 1). Reaches were
first broken into distinct geomorphic channel unit types following
Hawkins (1988), then transects with 0.2 m point spacing were placed at
the midpoint of each channel unit perpendicular to flow. At each point,
we measured total water depth and velocity at 60\% below the surface
using an acoustic doppler velocimeter (ADV - Flow Tracker 2, Sontek
Corporation). We visually classified substrate size into one of 5 size
classes following a Wentworth classification scale (Wentworth 1922).
Cover was estimated as present if structurally complex elements (e.g.,
wood) or overhanging vegetation was observed within the transect cell
(0.2 m to the left and right of velocity and depth measurement). Because
many fish were observed hiding in interstitial spaces underneath larger
substrate in snorkel surveys, we considered substrate over 25 cm
(substrate class 4 and above) as cover. Estimating cover is inherently
subjective and our classification likely missed nuanced differences
among different structural elements; however, it captured broad
differences between habitats where fish would be conspicuous and
habitats that would likely provide refuge from predators.

To quantify the role of diel variation in prey abundance, we measured
invertebrate drift during the day and at dusk in each reach on the same
day as snorkel observations of fish. Three 250 \(\mu\)M drift nets (306
cm\(^2\)) were set at the most upstream riffle habitat in each reach,
and left for two replicate 30 minute sets. We measured velocity and
depth at the opening of each net at the beginning and end of each set,
then computed the total volume filtered. Day sampling occurred between
10h00 and 14h00 and dusk sampling occurred within the hour around civil
twilight. Drift samples (\emph{n} = 3 per reach) were stored in 95\%
ethanol, then sorted in the laboratory, identified to family or order,
and measured to the nearest 0.1 mm using an ocular micrometer. We used
published regression equations to relate body lengths to biomass (Benke
et al. 1999).

We characterized rainbow trout and bull trout activity and microhabitat
use patterns by snorkeling. Fish were located by an observer, moving
upstream from the lower end of each reach. A dive flashlight was
necessary to locate fish during dusk and night surveys, and to search
under substrate and wood for hiding fish during the day. Once located,
individual fish were observed for 1-3 minutes and their activity was
recorded as either ``resting'', if they were motionless on the
streambed, or ``foraging'', if they held a focal position in the water
column to scan for invertebrate drift. In rare cases, activity was
ambiguous and these observations were not included in analysis. We
estimated body lengths of fish using marks placed on the observers
diving gloves as a reference. The timing of fish observations
corresponded to the day and dusk windows of drift sampling; night
observations started at least two hours after civil twilight.

In total, we observed the behaviour and microhabitat use of 192 bull
trout and 265 rainbow trout across the five reaches. Bull trout were
larger than rainbow trout on average (bull trout: 83 \(\pm\) 25;
rainbow: 62 \(\pm\) 25 mm), and the average size of both species
increased from day (bull trout: 68 \(\pm\) 14 mm; rainbow: 50 \(\pm\) 12
mm), to dusk (bull trout: 88 \(\pm\) 25 mm; rainbow: 62 \(\pm\) 23 mm)
to night (bull trout: 91 \(\pm\) 28 mm; rainbow: 79 \(\pm\) 30 mm)
observations. After behaviour was assigned and length was estimated, the
exact location of the fish was marked with a large washer for subsequent
measurements of microhabitat features; these included water depth,
velocity at the focal point of the fish and at 60\% of the total water
depth, focal point depth (as a proportion of total water depth),
substrate size, and whether structural cover was present.

\emph{Cover manipulation experiment} - To determine if predation risk
was driving diel activity and habitat use patterns, and if
foraging-predation risk trade-offs varied between species, we measured
the responses of fish to experimental additions of structural cover. We
constructed three sided boxes (40 cm L x 15 cm W x 20 cm H) that were
open facing the stream bed and at the upstream and downstream ends, with
natural branches attached on the underside to simulate structural cover
from wood (Figure 1). Boxes were oriented with the closed sides parallel
to flow to minimize impacts on local hydraulics (Appendix 2) and
anchored to the stream bottom with rebar stakes. We left boxes
undisturbed for four days, then returned to each box and observed
whether fish were present and actively foraging. Species and size of
each observed fish was estimated as described above in the snorkel
survey methods.

Our placement of boxes was designed to encompass a range of hydraulic
conditions, including: (1) areas where we observed fish foraging at dusk
but not during the day; (2) apparently unsuitable drift-foraging areas
with zero velocity; and (3) unsuitable foraging areas with high velocity
( \textgreater{} 60 cm s \(^{-1}\)). We hypothesized that if predation
risk was a key factor restricting diurnal foraging, cover added to more
suitable foraging locations should be colonized by daytime foraging
fish. No fish were observed foraging during the day at any cover box
sites prior to their placement.

Because the suitability of foraging habitat is influenced by food
abundance in addition to velocity and depth, we combined measurements of
velocity with stream-averaged daytime invertebrate drift concentrations
to estimate the potential net rate of energy intake (NREI Joules
s\(^{-1}\)) in each location where cover was added (Appendix 2). NREI is
defined as the total rate of energy gained from drift foraging less the
costs of swimming, maneuvering, and basic metabolic demands (Hughes and
Dill 1990; Piccolo et al. 2014). We implemented the drift-foraging
bioenergetics model using the freeware program BioenergeticHSC
(described in Naman et al. 2020).

\emph{Statistical analysis}

\emph{Diel activity and habitat use patterns}- All statistical analyses
were conducted in R version 4.0.2 (R Core Development Team). To
determine whether activity (foraging vs.~resting) differed for each
species through time and between species within a given time period, we
used Z-tests of equal proportions or Fisher exact probability test (when
\emph{n} \textless{} 5 observations). We then used a combination of
univariate and multivariate approaches to examine differences in
microhabitat use between species (rainbow vs.~bull trout) and activities
(foraging vs.~resting) throughout the diel cycle. To examine how species
differed in depth and velocity microhabitat use, we compared focal point
velocity, water depth, and focal depth across all combinations of
species and activity in each time period (day, dusk night) using
Kruskal-Wallis rank sum tests because parametric assumptions could not
be met. Where Kruskal-Wallis tests were significant (\emph{P}
\textless{} 0.05), we used pairwise Wilcoxon rank-sum tests to identify
what activity by species drove the differences. A Holm correction factor
was applied to account for inflated type 1 error with multiple
comparisons.

To evaluate microhabitat overlap in multivariate space, we conducted a
Canonical Analysis of Principal Coordinates (CAP; Anderson and Willis
2003). CAP is a constrained ordination procedure designed to uncover
patterns in multivariate data where there are \emph{a priori} hypotheses
about differences among groups. For each time period, we determined the
extent that microhabitat features used by fish diverged according to
species and activity. Microhabitat variables included focal velocity,
water depth, focal depth (as a proportion of water depth), average water
column velocity, cover presence, and relative substrate size. We
performed the CAP analysis on a Gower similarity matrix of scaled
habitat variables using the \emph{capscale} function from the vegan
package in R (Oksanen et al. 2013). We then tested the significance of
constrained axes using permutation tests (\emph{anova.cca} function in
vegan).

Body size can be another important determinant of diel habitat use and
activity. However, we did not incorporate body length into our formal
statistical analysis for several reasons: first, we did not have
information on length-age relationships in the Skagit system and
exploratory size-frequency plots did not reveal clear age classes; and
second, creating length class bins would have reduced statistical power
due to low numbers of observations. Therefore, we consider bull trout
and rainbow trout as a single assemblage of juvenile fish, but recognize
that body size differences may contribute to observed variation.

\emph{Cover manipulation} - Predation risk and hydraulic habitat quality
may act as hierarchical habitat filters determining microhabitat
occupancy by fish. We tested whether the probability of diurnal cover
box occupancy by foraging fish increased in more energetically
profitable areas (i.e., higher NREI). We used a generalized linear model
(GLM) with a binomial error distribution and a logit link function to
estimate the probability of occupancy (scaled 0 to 1) as a function of
NREI, then applied a likelihood ratio test to compare the effect of NREI
to a reduced model with only an intercept term.

Colonization of cover boxes could occur by two distinct pathways: (1)
nocturnal and crepuscular foragers becoming day-active; or (2) diurnal
foragers redistributing to forage under the added cover. Because size
distributions varied with time of day, they can potentially inform these
alternate mechanisms. Similar size distributions between fish using
added cover and fish observed at dusk would support altered diel
activity patterns; alternatively, similar size distributions between
fish using added cover and fish observed during the day would support
re-distribution. We used two-sample Kolmogorov-Smirnov tests to compare
size distributions of fish we observed foraging under added cover to
fish foraging at dusk vs.~during the day

\emph{Invertebrate drift abundance and size distribution} - We compared
the invertebrate drift abundance between day and dusk using one-way
analysis of variance. We also examined drift size distributions to test
the prediction that drift at dusk would have more large invertebrates,
which has a strong bearing on the energetic profitability of foraging
(Dodrill et al. 2016). We fit gamma distributions to pooled drift data
from each time period using the R package \emph{fitrdistplus}. Rate
(\(\beta\)) and shape (\(\alpha\)) parameters from the gamma
distributions (\(g(x|\beta,\alpha)\)) were then bootstrapped and used to
make inferences into differences in drift size distributions.
Specifically, we compared bootstrapped 95\% confidence intervals of the
mean size (\(\alpha/\beta\)) and the skewness (\(2/\sqrt(\alpha)\)),
where a greater negative skewness value indicates a larger proportion of
larger individuals.

\emph{Habitat suitability modelling} - To investigate how diel changes
in microhabitat use patterns influence habitat suitability models, we
developed diurnal, crepuscular, and nocturnal univariate habitat
suitability curves (HSCs) for water depth, average water column
velocity, substrate, and cover presence. HSCs are widely used indices of
habitat quality based on ratios of habitat use to habitat availability,
standardized between 0 and 1 (Nestler et al. 2019). We pooled habitat
use frequencies across all individuals for a given species and time of
day. Similarly, we pooled habitat availability from transect data across
all reaches, weighted by the relative area each transect represented.
This approach assumes that our habitat sampling is representative of the
conditions available to fish in the Skagit system.

For each species and time period, we used binomial GLMs with a logit
link function to generate HSCs by relating the availability of a given
habitat feature to the relative probability of habitat use. This method
is analogous to resource selection functions based on logistic
regression (Ayllón et al. 2012). We tested linear and second order
polynomial terms for each habitat variable and retained the most
parsimonious model based on Akiake's Information Criterion (AIC)
adjusted for small sample sizes (Burnham and Anderson 2002). Model
predictions were then rescaled to a maximum of 1. Depth and velocity
data were binned into intervals of 5 cm and 5 cm s\(^{-1}\) for this
analysis. This univariate HSC method effectively ignores model
prediction uncertainty as well as potential interactions among habitat
features. However, our goal was to replicate univariate HSCs that are
typically developed and used by practitioners (Nestler et al. 2019), and
to consider the implications of qualitative differences in the shape of
HSCs among time periods and between species. Thus, we did not pursue a
more rigorous statistical approach.

Due to limited sample size, we were not able to directly calculate
separate HSCs for foraging vs.~resting fish. However, because the of
strong temporal trend in activity patterns (see \emph{Results}), HSCs
compared across time periods largely reflect differences between these
contrasting behaviours. While we report habitat suitability results for
substrate, and cover, we focus our inference and discussion on HSCs for
velocity and depth.

\section{Results}\label{results}

\emph{Patterns of activity and microhabitat use} - Both rainbow trout
and bull trout activity exhibited striking diel patterns, with a
significantly higher proportion of fish foraging at dusk relative to day
or night (Figure 2; rainbow:, \(\chi^2\) = 95.682, \emph{P} \textless{}
0.001; bull trout: \(\chi^2\) = 69.448, \emph{P} \textless{} 0.001).
Species differences were also evident, but were not as pronounced as
differences among time periods. During the day 61\% of rainbow trout
were foraging relative to only 25\% of bull trout (\(\chi^2\) = 15.63,
\emph{P} \textless{} 0.001), and at dusk 97\% of rainbow trout were
foraging relative to 81\% of bull trout (\(\chi^2\) = 9.89, \emph{P} =
0.002). Both species switched to primarily resting behaviour at night,
and contrary to expectations, there were no differences in the
proportion of foraging vs.~resting fish between species (rainbow = 24\%
foraging; bull trout = 16\% foraging, \(\chi^2\) = 1.13, \emph{P} =
0.29).

Microhabitat use also exhibited diel variation; in particular, there was
a strong crepuscular shift towards decreased use of cover by both
species (rainbow: \(\chi^2\) = 91.39, \emph{P} \textless{} 0.001; bull
trout: \(\chi^2\) = 64.57, \emph{P} \textless{} 0.001). 86\% of rainbow
trout and 100\% of bull trout were associated with cover in daylight,
then at dusk cover use decreased to 19\% of rainbow trout and 32\% of
bull trout. Nocturnal cover use remained lower than in daylight but was
slightly elevated relative to dusk (35\% of rainbow trout and 42\% of
bull trout) when species were mostly drift-feeding. Species contrasts in
cover were only statistically different during daylight, when 14\% fewer
rainbow trout were associated with cover than bull trout (Fisher exact
test: \emph{P} = 0.002; dusk: \(\chi^2\)= 2.89, \emph{P} = 0.09; night:
\(\chi^2\) = 0.45, \emph{P} = 0.50).

Daytime velocity use differed with activity (\(\chi^2\) = 75.862,
\emph{P} \textless{} 0.001), with foraging rainbow and bull trout
occupying faster velocities than their resting conspecifics (Figure 3;
Wilcox rank sum test: bull trout foraging vs.~resting \emph{P}
\textless{} 0.001; rainbow trout foraging vs.~resting \emph{P}
\textless{} 0.001); however, focal velocities did not differ between
species for a given activity (Wilcox rank sum test; foraging \emph{P} =
0.44, resting \emph{P} = 0.43). At dusk, foraging individuals continued
to occupy faster velocities than resting individuals (\(\chi^2\) =
23.292, \emph{P} \textless{} 0.001), and small differences in focal
foraging velocities emerged between species, with foraging rainbow trout
occupying faster velocities than bull trout (\emph{P} = 0.043). This
difference did not persist at night; although foraging fish still
occupied faster velocities than resting individuals (\(\chi^2\) =
28.106, \emph{P} \textless{} 0.001; rainbow foraging vs.~resting
\emph{P} = 0.002; bull trout foraging vs.~resting \emph{P} = 0.001).

Depth use differed more between species than activities. During the day,
differences in depth use (\(\chi^2\) = 20.565, \emph{P} \textless{}
0.001) were driven by foraging bull trout foraging in deeper habitats
than other groups (\emph{P} \textless{} 0.001 across all pairwise
comparisons). At dusk, bull trout continued to occupy deeper habitats
than rainbow trout (\(\chi^2\) = 25.387, \emph{P} \textless{} 0.001),
but there were no differences in depth use between resting and foraging
fish (\emph{P} = 0.98). At night, differences were evident across
species and activities (\(\chi^2\) = 22.926, \emph{P} \textless{}
0.001), with bull trout generally occupying deeper habitat (\emph{P} =
0.001).

Multivariate CAP analysis further supported divergent microhabitat use
between species and activities (Figure 4); however, the overall variance
explained by the constrained axes was low (Day: 26\%; Dusk: 8\%; Night:
15\%). During the day, the primary axis of separation (explaining 88\%
of the constrained variance) was driven by increasing focal velocity and
decreasing average velocity, focal depth, and substrate (species: SS =
0.19, \emph{F} = 6.16, \emph{P} \textless{} 0.001; activity: SS = 1.09,
\emph{F} = 35.03, \emph{P} \textless{} 0.001), indicating a contrast in
microhabitat selection between foraging and resting fish. The second
axis explained 11\% of the constrained variance and was driven by
increasing water depth and cover presence, which tended to separate bull
trout from rainbow trout. At dusk, the overwhelming majority of
constrained variance (93\%) was driven by increasing total depth, focal
depth, and cover, and decreasing substrate size and focal and average
velocity (species: SS = 0.55, \emph{F} = 9.84, \emph{P} \textless{}
0.001; activity: SS = 0.12, \emph{F} = 2.18, \emph{P} = 0.048), which
again separated bull trout (deeper, lower velocity) from rainbow trout
(shallower, higher velocity). At night, the primary axis explained 81\%
of the constrained variation and was driven by increasing velocity
(focal and average) and decreasing focal depth that differentiated
foraging and resting fish. The second axis explained 19\% of the
constrained variation and was driven by increasing substrate size and
decreasing total depth (species: SS = 0.19, \emph{F} = 4.26, \emph{P} =
0.004; activity: SS = 0.83, \emph{F} = 18.69, \emph{P} \textless{}
0.001), again reflecting differences in resting habitat selection
between bull trout (deeper) and rainbow trout (shallower).

Irrespective of time of day, resting individuals of both species were
associated with lower focal velocities than foraging individuals.
However, the contribution of other habitat attributes was dynamic. For
instance, foraging microhabitats were associated with deeper water
depths and smaller substrate size during the day, then shifted to
shallow depths and larger substrate at dusk (Figure 4). Species
contrasts were more pronounced when foraging relative to resting, with
foraging rainbow trout occupying faster, shallower habitats than
foraging bull trout, but differences were no longer apparent when fish
were at rest.

\emph{Cover addition experiment: predation risk and food abundance} - In
line with our predictions, cover boxes that were placed in more suitable
foraging velocities, as determined by NREI, had a higher probability of
being occupied by rainbow trout (Likelihood Ratio Test \(\chi^2\) =
11.397, Deviance = 33.27, \emph{P} = 0.001; Figure 5).
Kolmogorov-Smirnov tests indicated that fish foraging in the added cover
were smaller than those foraging fish at dusk (Figure S3; \emph{D} =
0.398, \emph{P} = 0.02) and similar to those foraging during the day
(\emph{D} = 0.19, \emph{P} = 0.70). No bull trout of any size colonized
the cover boxes.

Also in line with predictions, invertebrate drift concentration
increased at dusk relative to day (Figure S4). The number of drifting
invertebrates increased at dusk from 5.4 m\(^{-3}\) to 22.5\(^{-3}\)
(ANOVA \emph{F} = 20.82, \emph{P} = 0.001; 95\% CIs: day = 0.28-10.65;
dusk = 15.99-29.10) and the biomass concentration of drifting
invertebrates increased from 0.8 mg m\(^{-3}\) to 4.6 m\(^{-3}\)
(\emph{F} = 6.82, \emph{P} = 0.02; 95\% CIs: day = 0-2.79; dusk =
2.12-7.02). In addition, invertebrates drifting at dusk were larger on
average (mean day: 1.7 mm, 95\% CIs = 1.6-1.8 mm; mean dusk: 2.0 mm,
95\% CIs = 1.9-2.1 mm) and were less skewed towards smaller individuals
(skewness day: 1.12, 95\% CIs = 1.09-1.13; dusk: 1.09, 95\% CIs =
1.08-1.1).

\emph{Effects of diel behaviour on habitat suitability criteria} -
Univariate HSCs exhibited marked differences across time periods and
species (Figure 6). For rainbow trout, optimal velocities shifted from 0
m s\(^{-1}\) during the day, when fish were concealed under low velocity
cover, to 0.45 m s\(^{-1}\) at dusk when fish were primarily
drift-foraging in the open. At night, optimal velocities declined to
0.25 m s\(^{-1}\), reflecting a mix of activities. By contrast, optimal
velocities for bull trout were similar between day and night (0.25 and
0.3 m s\(^{-1}\)) but shifted to 0 m s\(^{-1}\) at dusk. However, bull
trout HSCs should be tempered by low sample sizes. Depth HSCs for both
species shifted from a linear increase in suitability with depth during
the day, to distinct peaks at dusk and at night (Figure 6). Optimal dusk
and night depths were similar for rainbow trout but optimal depths for
bull trout declined from dusk (1.1 m) to night (0.65 m). HSCs for
substrate and cover reflected the observed diel shifts in behaviour and
exhibited contrasts between species. For rainbow trout, microhabitats
with cover had higher suitability values during the day, but lower
values at night (Figure 6). Alternatively, there was less diel
variability in cover suitability for bull trout, except for the strong
increase in suitability for `No cover' microhabitats from daylight to
dusk and night. There was also diel variation in substrate suitability.
Bull trout shifted from larger substrate during the day to smaller
substrate at dusk and night. Directional patterns were less clear for
rainbow trout (Figure 6).

\section{Discussion}\label{discussion}

\emph{Diel activity and habitat use patterns}

Rainbow trout and bull trout in the Skagit River exhibited striking and
broadly consistent diel activity patterns, where foraging was
predominately crepuscular. While not universal, this is a common pattern
observed in stream salmonids (e.g., Alanärä and Brännäs 1997; Sato and
Watanabe 2014) that is supported by theory relating diel activity to
foraging-predation risk trade-offs (Railsback et al. 2020a). These
investigations suggest that elevated prey abundance, moderate foraging
efficiency, and reduced predation risk during crepuscular periods offers
salmonids a positive balance of risk and reward. Our results support
this interpretation. At dusk, drifting invertebrates were larger and
more abundant, and the majority of fish were not associated with
structural cover, suggesting predation risk was no longer a strong
constraint on habitat selection. Temporal shifts in activity patterns
were paralleled by shifts in microhabitat use, reflecting the dynamic
suite of habitat conditions required for feeding, resting, and predator
avoidance. These include structural cover in slower stream margins
during the day, faster velocities that are energetically profitable for
drift-foraging at dusk, and slower resting velocities at night.

Despite broadly similar activity and habitat use patterns, nuanced
differences were still evident between the two species. First, bull
trout foraged less frequently during the day than rainbow trout, which
aligns with previous work on these taxa (Angradi and Griffith 1990;
Baxter and McPhail 1997). However, there was no evidence that bull trout
foraged more frequently than rainbow trout at dusk or at night, i.e.,
there was a higher proportion of rainbow trout foraging in all time
periods. This could suggest that juvenile bull trout allocate less
effort to foraging than rainbow trout due divergence in physiology or
life history traits (discussed further below). Alternatively, it could
reflect bull trout using other foraging modes, such as benthic or
interstitial search foraging (Nakano et al. 1992) that our snapshot
observations did not capture.

Second, multivariate analyses illuminated subtle differences in
microhabitat use, with rainbow trout foraging in faster, shallower areas
than bull trout during both day and at dusk. Rainbow trout using higher
velocities than other species is generally consistent with previous
studies (Bisson et al. 1988), as is the association of bull trout with
deeper slower microhabitats (Polacek and James 2003; Al-Chokhachy and
Budy 2007). By contrast, microhabitat partitioning was less pronounced
among resting individuals, which generally presented as a mixed
assemblage associated with lower velocities.

Collectively, these results provide only limited support for the role of
temporal resource partitioning as a strong mechanism of rainbow trout
and bull trout coexistence in the upper Skagit system. Instead, our
findings suggests that these species are dealing with broadly similar
constraints to their activity and habitat use patterns, and that
differentiation is nuanced and potentially cryptic (see discussion of
cover experiment below). Further, the extent of habitat overlap we
observed suggests these species may compete for foraging, resting, or
refuging habitat (Fausch and White 1981; Nakano et al. 1999). However,
since our study was observational we cannot make direct inferences
concerning competition, and additional work is necessary to investigate
this possibility.

\emph{Experimental cover additions and potential mechanisms of adaptive
differentiation}

Experimental cover additions illuminated the influence of predation risk
on diel habitat use patterns, as well as key differences between
species. Colonization of cover boxes by rainbow trout was consistent
with the hypothesis that predation risk is a key driver of diel activity
and habitat shifts, limiting daytime foraging away from cover. The
ability of NREI to predict the colonization probability of a given cover
box further reinforces this inference. In essence, this finding
highlights the role of predation risk as a constraint on diurnal habitat
use and the presence of structural cover as a key dimension of habitat
quality (Boss and Richardson 2002; Kawai et al. 2014; but see Larranaga
and Steingrímsson 2015).

Colonization of cover boxes by smaller fish indicates redistribution of
smaller diurnal foragers to more profitable locations, rather than
shifts in diel activity patterns (i.e., larger fish shifting from
crepuscular to diurnal foraging). The lack of response by larger fish to
cover additions could relate to additional diurnal foraging not being
sufficiently beneficial for larger individuals. For instance, if fish
can meet their metabolic demands during crepuscular periods when food is
more abundant, foraging under the added cover in daylight would have
limited benefit. Another possibility is that the added cover may not
have sufficiently reduced the threat of predation for larger fish that
are often more risk averse (Naman et al. 2019b). We are unable to
disentangle these mechanisms; however, on the whole these results align
with theory and observations of size-dependent diel foraging patterns
(Sato and Watanabe 2014; Railsback et al. 2020a).

In contrast to rainbow trout, no bull trout of any size colonized any of
the cover boxes. This striking result cannot be explained by differences
in body size between species, given that many bull trout observed during
the day were within the size range of rainbow trout that colonized the
added cover. Interspecific competition is also an unlikely explanation,
given that \textasciitilde{}30\% of cover boxes remained unoccupied
despite being in energetically profitable locations (NREI \textgreater{}
0). Instead, we posit that these contrasting responses relate to
divergent traits between these species that modify foraging-predation
risk trade-offs and how individuals respond to them.

Adaptive divergence in juvenile salmonids is best understood in terms of
the integrated phenotype, where niche differentiation results in a suite
of integrated physiological, behavioural, and morphological traits that
determine species performance in different habitats (Monnet et al. 2020;
Rosenfeld et al. 2020). For example, Rosenfeld et al. (2020) showed that
sympatric juvenile rainbow trout and coho salmon are adaptively
differentiated along and axis of high food consumption and low growth
efficiency (trout) vs.~lower food consumption and high growth efficiency
(coho salmon). Similarly, there may be a number of adaptive trait
combinations that underlie adaptive differentiation between rainbow
trout and bull trout; here we discuss two particularly intriguing
possibilities. First, species in the genus \emph{Salvelinus} (including
bull trout) appear to feed more efficiently at low light levels than
species in the genus \emph{Oncorhynchus} (including rainbow trout)
(Elliott 2011). Consequently, bull trout may receive a lower marginal
benefit from the diurnal foraging opportunities provided by the added
cover. More detailed observations of foraging efficiency or stomach
fullness would shed further light on this possibility. Second,
fundamental differences in risk aversion may stem from contrasting life
history strategies between the species. While there is significant
variability, these species (and \emph{Oncorynchus} and \emph{Salvelinus}
more generally) diverge across a growth-longevity axis, with rainbow
trout tending to exhibit earlier age at maturity and faster early growth
relative to bull trout (Erhardt and Scarnecchia 2016; McCubbins et al.
2016; Quinn 2018). As a result, juvenile rainbow may exhibit integrated
traits associated with a `faster' pace of life syndrome (Réale et al.
2010), which includes higher prey consumption and bolder risk-taking
behaviour driven by higher metabolic demands (Tyler and Bolduc 2008;
Mesa et al. 2013; Monnet et al. 2020). By contrast, juvenile bull trout
may exhibit a more risk averse strategy and grow slowly before
transitioning into piscivory as sub-adults (Rieman and McIntyre 1993).
Additional life history information on these species would be valuable
in the Skagit, which likely includes both stream-resident and adfluvial
population components.

While our observation of crepuscular foraging by bull trout is
consistent with their known behaviour, not all rainbow trout populations
are predominantly crepuscular. For example, juvenile anadromous rainbow
trout (steelhead) and coho salmon were observed to actively forage
throughout the day away from cover in nearby coastal watersheds with
similar food abundance as the Skagit (Naman et al. 2019a). This suggests
qualitative differences in predation risk between systems. One
possibility is elevated predation risk associated with the presence of
larger piscivorous bull trout, which prey on younger age classes (Pinto
et al. 2013) and may be active in daylight (Eckmann et al. 2018).
Crepuscular foraging in the presence of adult bull trout may be a
flexible adaptation in sympatric juvenile rainbow trout but a fixed one
in juvenile bull trout that always experience high risk from older age
classes. Alternatively, predation risk in the presence of larger
resident trout may restrict juvenile daytime foraging irrespective of
species (Walters and Juanes 1993), while remaining low in anadromous
streams that often lack larger resident piscivores. A more systematic
assessment of rainbow trout diel foraging behaviour in anadromous
vs.~resident streams is required to properly evaluate this hypothesis.

We acknowledge that our insights into these potential mechanisms of
adaptive differentiation are speculative given the limited information
available in the upper Skagit system and the wide variation in
behavioural, physiological, and life history traits \emph{within}
salmonid species (e.g., McMillan et al. 2012). Nonetheless, our results
highlight the importance of further understanding the cryptic axes of
differentiation among sympatric stream salmonids; in particular,
divergent responses to variability in habitat conditions (Bradford and
Higgins 2001), predation risk (Harvey and Nakamoto 2013, this study), or
resource abundance (Bailey and Moore 2020) that collectively define the
integrated phenotype.

\emph{Implications for habitat suitability modelling}

The observation that salmonid microhabitat use varies throughout the day
is not new, yet many habitat suitability models for salmonids are based
only on daytime observations (Nestler et al. 2019; Railsback et al.
2020b). Our qualitative comparisons of HSCs generated in distinct diel
phases highlight potential issues that may arise with species that show
temporal variation in habitat use. Most notably, the peaks of velocity
HSCs for rainbow trout shifted from near 0 m s\(^{-1}\) during the day
to \textasciitilde{}0.5 m s \(^{-1}\) at dusk, when the majority of
individuals were drift-feeding. This suggests that suitability criteria
generated from daylight observations will poorly represent the habitat
conditions required for energy acquisition for fish populations that
refuge during the day due to high predation risk. Consequently, these
populations may be more sensitive to altered velocity distributions
(e.g., through flow reduction) than daytime habitat suitability
predictions would suggest (Railsback et al. 2020b). For example, a diel
HSC with a peak at 0 m s\(^{-1}\) would predict that very low velocities
(and therefore flows) would be optimal for fish production. However,
flows that maximize zero velocity microhabitats that trout use during
the day in the Skagit would severely reduce the area of high energy flux
microhabitats necessary for trout to maximize energy intake during
crepuscular foraging.

A further challenge is that diel habitat use patterns are variable
across diverse conditions. For example, the extent of diurnal foraging
is often strongly related to temperature, which drives metabolic demands
(Fraser et al. 1993). Similarly, the magnitude of risk from different
terrestrial and aquatic predators varies seasonally and across systems
(Harvey and Nakamoto 2013). These conditions, along with
inter-individual variation, will ultimately determine the extent of diel
habitat shifts and the consequences of developing and applying
fish-habitat relationships across time periods. Borrowing
literature-derived HSCs may therefore introduce significant error; for
example, borrowing HSCs with unknown predation risk may be inappropriate
if they are based on daytime refuging behavior that poorly represents
foraging velocity selection.

How can habitat suitability assessments account for this complexity?
Unlike individual based modelling approaches, habitat suitability models
based on frequency-of-use observations cannot deal with the mechanistic
drivers of habitat selection explicitly. However, a basic approach could
be to observe fish across the full diel cycle, then generate separate
HSCs from different times. These HSCs would then define a range of
habitat suitability predictions to inform management or be used for
sensitivity analysis (Railsback et al. 2020b). For example, juvenile
salmonid instream flow requirements in the Skagit could be
conservatively based on crepuscular velocity requirements that maximize
feeding opportunities, since foraging is essential for growth and the
most flow-demanding behaviour. While this approach would require
additional field effort, it would likely reduce the potential bias
associated with predation risk and diel variation in habitat use.

\section{Acknowledgements}\label{acknowledgements}

We are grateful to the Skagit Environmental Endowment Commission and the
Natural Sciences and Engineering Research Council for supporting this
work. Danielle Courcelles, Ashley Rawhauser, Rob Wilson, Kevin Wilson
provided helpful advice on study design and logistics. Paige Lewis and
Thomas Smith helped with field data collection and Alyssa Nonis assisted
with invertebrate sorting. Scott Denkers and Chris Tunnoch also provided
invaluable advice and support in initiating this project. Two anonymous
reviewers provided helpful comments that improved the manuscript.

\section{Contributions}\label{contributions}

SN and JR conceived and designed the study; SN, AL, and JR conducted the
fieldwork; SN analyzed the data and wrote the initial draft of the
manuscript; and all authors contributed to revisions.

\section{Data Availability}\label{data-availability}

Data and R code supporting this manuscript are available at
\url{https://github.com/seannaman/Naman-et-al_2021_CJFAS.git}.

\section{References}\label{references}

\hypertarget{refs}{}
\hypertarget{ref-Al-Chokhachy2007a}{}
Al-Chokhachy, R., and Budy, P. 2007. Summer Microhabitat Use of Fluvial
Bull Trout in Eastern Oregon Streams. North American Journal of
Fisheries Management \textbf{27}(4): 1068--1081.
doi:\href{https://doi.org/10.1577/m06-154.1}{10.1577/m06-154.1}.

\hypertarget{ref-Alanara1997}{}
Alanärä, A., and Brännäs, E. 1997. Diurnal and nocturnal feeding
activity in Arctic char (Salvelinus alpinus) and rainbow trout
(Oncorhynchus mykiss). Canadian Journal of Fisheries and Aquatic
Sciences \textbf{54}(12): 2894--2900.
doi:\href{https://doi.org/10.1139/f97-187}{10.1139/f97-187}.

\hypertarget{ref-Anderson2003}{}
Anderson, M.J., and Willis, T.J. 2003. Canonical analysis of principal
coordinates: A useful method of constrained ordination for ecology.
Ecology \textbf{84}(2): 511--525.
doi:\href{https://doi.org/10.1890/0012-9658(2003)084\%5B0511:CAOPCA\%5D2.0.CO;2}{10.1890/0012-9658(2003)084{[}0511:CAOPCA{]}2.0.CO;2}.

\hypertarget{ref-Angradi1990}{}
Angradi, T.R., and Griffith, J.S. 1990. Diel feeding chronology and diet
selection of rainbow trout (Oncorhynchus mykiss) in the Henry's Fork of
the Snake River, Idaho. Canadian Journal of Fisheries and Aquatic
Sciences \textbf{47}(1): 199--209.
doi:\href{https://doi.org/10.1139/f90-022}{10.1139/f90-022}.

\hypertarget{ref-Ayllon2012a}{}
Ayllón, D., Almodóvar, A., Nicola, G.G., and Elvira, B. 2012. The
influence of variable habitat suitability criteria on PHABSIM habitat
index results. River Research and Applications \textbf{28}(8):
1179--1188. doi:\href{https://doi.org/10.1002/rra}{10.1002/rra}.

\hypertarget{ref-Bailey2020}{}
Bailey, C.J., and Moore, J.W. 2020. Resource pulses increase the
diversity of successful competitors in a multi-species stream fish
assemblage. Ecosphere \textbf{11}(September).
doi:\href{https://doi.org/10.1002/ecs2.3211}{10.1002/ecs2.3211}.

\hypertarget{ref-Baxter1997a}{}
Baxter, J.S., and McPhail, J.D. 1997. Diel microhabitat preferences of
juvenile bull trout in an artificial stream channel. North American
Journal of Fisheries Management \textbf{17}(4): 975--980.
doi:\href{https://doi.org/10.1577/1548-8675(1997)017\%3C0975:dmpojb\%3E2.3.co;2}{10.1577/1548-8675(1997)017\textless{}0975:dmpojb\textgreater{}2.3.co;2}.

\hypertarget{ref-Benke1999}{}
Benke, A.C., Huryn, A.D., Smock, L.A., and Wallace, J.B. 1999.
Length-mass relationships for freshwater macroinvertebrates in North
America with particular reference to the southeastern United States.
Journal of the North American Benthological Society \textbf{18}(3):
308--343.

\hypertarget{ref-Bishop1969a}{}
Bishop, J.E. 1969. Light control of aquatic insect activity and drift.
Ecology \textbf{50}(3): 371--380.

\hypertarget{ref-Bisson1988}{}
Bisson, P.A., Sullivan, K., and Nielsen, J.L. 1988. Channel hydraulics,
habitat use, and body form of juvenile coho salmon, steelhead, and
cutthroat trout in streams. Transactions of the American Fisheries
Society \textbf{117}: 262--273.

\hypertarget{ref-Boss2002}{}
Boss, S.M., and Richardson, J.S. 2002. Effects of food and cover on the
growth, survival, and movement of cutthroat trout (Oncorhynchus clarki)
in coastal streams. Canadian Journal of Fisheries and Aquatic Sciences
\textbf{59}: 1044--1053.
doi:\href{https://doi.org/10.1139/F02-079}{10.1139/F02-079}.

\hypertarget{ref-Bradford2001}{}
Bradford, M.J., and Higgins, P.S. 2001. Habitat-, season-, and
size-specific variation in diel activity patterns of juvenile chinook
salmon (Oncorhynchus tshawytscha) and steelhead trout (Oncorhynchus
mykiss). Canadian Journal of Fisheries and Aquatic Sciences (58):
365--374.
doi:\href{https://doi.org/10.1139/cjfas-58-2-365}{10.1139/cjfas-58-2-365}.

\hypertarget{ref-Burnham2002}{}
Burnham, K.P., and Anderson, D.R. 2002. Model selection and multi-model
inference: a practical information theoretic approach. Springer Science;
Buisiness Media.

\hypertarget{ref-Dodrill2016b}{}
Dodrill, M.J., Yackulic, C.B., Kennedy, T.A., and Hayes, J.W. 2016. Prey
size and availability limits maximum size of rainbow trout in a large
tailwater: insights from a drift-foraging bioenergetics model. Canadian
Journal of Fisheries and Aquatic Sciences \textbf{73}: 759--772.
doi:\href{https://doi.org/10.1139/cjfas-2015-0268}{10.1139/cjfas-2015-0268}.

\hypertarget{ref-Eckman2016}{}
Eckmann, M., Dunham, J., Connor, E.J., and Welch, C.A. 2018.
Bioenergetic evaluation of diel vertical migration by bull trout
(Salvelinus confluentus) in a thermally stratified reservoir. Ecology of
Freshwater Fish \textbf{27}(1): 30--43.
doi:\href{https://doi.org/10.1111/eff.12321}{10.1111/eff.12321}.

\hypertarget{ref-Elliott1973}{}
Elliott, J.M. 1973. The food of brown and rainbow trout (Salmo trutta
and S. gairdneri) in relation to the abundance of drifting invertebrates
in a mountain stream. Oecologia \textbf{12}(4): 329--347.
doi:\href{https://doi.org/10.1007/BF00345047}{10.1007/BF00345047}.

\hypertarget{ref-Elliott2011}{}
Elliott, J.M. 2011. A comparative study of the relationship between
light intensity and feeding ability in brown trout (Salmo trutta) and
Arctic charr (Salvelinus alpinus). Freshwater Biology \textbf{56}(10):
1962--1972.
doi:\href{https://doi.org/10.1111/j.1365-2427.2011.02627.x}{10.1111/j.1365-2427.2011.02627.x}.

\hypertarget{ref-Erhardt2016}{}
Erhardt, J.M., and Scarnecchia, D.L. 2016. Growth model selection and
its application for characterizing life history of a migratory bull
trout (Salvelinus confluentus) population. Northwest Science
\textbf{90}(3): 328--339.
doi:\href{https://doi.org/10.3955/046.090.0311}{10.3955/046.090.0311}.

\hypertarget{ref-FauschWhite1981}{}
Fausch, K.D., and White, R.J. 1981. Competition between brook trout
(Salvelinus fontinalis) and brown trout (Salmo trutta) for positions in
a Michigan stream. Canadian Journal of Fisheries and Aquatic Sciences
\textbf{38}(10): 1220--1227.
doi:\href{https://doi.org/10.1139/f81-164}{10.1139/f81-164}.

\hypertarget{ref-Fraser1997}{}
Fraser, N.H.C., and Metcalfe, N.B. 1997. The costs of becoming
nocturnal: feeding efficiency in relation to light intensity in juvenile
Atlantic Salmon. Functional Ecology \textbf{11}(3): 385--391.
doi:\href{https://doi.org/10.1046/j.1365-2435.1997.00098.x}{10.1046/j.1365-2435.1997.00098.x}.

\hypertarget{ref-Fraser1993}{}
Fraser, N.H.C., Metcalfe, N.B., and Thorpe, J.E. 1993.
Temperature-dependent switch between diurnal and nocturnal foraging in
salmon. Proceedings of the Royal Society B: Biological Sciences
\textbf{252}: 135--139.

\hypertarget{ref-Harris1992}{}
Harris, D., Hubert, W., and Wesche, T. 1992. Habitat use by
young-of-year brown trout and effects on weighted usable area. Rivers
\textbf{3}: 99--105.

\hypertarget{ref-Harvey2013}{}
Harvey, B.C., and Nakamoto, R.J. 2013. Seasonal and among-stream
variation in predator encounter rates for fish prey. Transactions of the
American Fisheries Society \textbf{142}(3): 621--627.
doi:\href{https://doi.org/10.1080/00028487.2012.760485}{10.1080/00028487.2012.760485}.

\hypertarget{ref-Harvey2016}{}
Harvey, B.C., and White, J.L. 2016. Use of cover for concealment
behavior by Rainbow Trout: Influences of cover structure and area. North
American Journal of Fisheries Management \textbf{36}(6): 1308--1314.
Taylor \& Francis.
doi:\href{https://doi.org/10.1080/02755947.2016.1207728}{10.1080/02755947.2016.1207728}.

\hypertarget{ref-Hearn1987}{}
Hearn, W.E. 1987. Interspecific competition and habitat segregation
among stream-dwelling trout and salmon: A review. Fisheries
\textbf{12}(5): 24--31.
doi:\href{https://doi.org/10.1577/1548-8446(1987)012\%3C0024:icahsa\%3E2.0.co;2}{10.1577/1548-8446(1987)012\textless{}0024:icahsa\textgreater{}2.0.co;2}.

\hypertarget{ref-Hughes1990b}{}
Hughes, N.F., and Dill, L.M. 1990. Position choice by drift feeding
salmonids: a model and test for arctic grayling (Thymallus arcticus) in
subarctic mountain streams, interior Alaska. Canadian Journal of
Fisheries and Aquatic Sciences \textbf{47}(1984): 2039--2048.

\hypertarget{ref-Jakober2000}{}
Jakober, M.J., McMahon, T.E., and Thurow, R.F. 2000. Diel habitat
partitioning by bull charr and cutthroat trout during fall and winter in
Rocky Mountain streams. Environmental Biology of Fishes \textbf{59}(1):
79--89.
doi:\href{https://doi.org/10.1023/A:1007699610247}{10.1023/A:1007699610247}.

\hypertarget{ref-Kawai2014}{}
Kawai, H., Nagayama, S., Urabe, H., Akasaka, T., and Nakamura, F. 2014.
Combining energetic profitability and cover effects to evaluate salmonid
habitat quality. Environmental Biology of Fishes (Fausch 1988).
doi:\href{https://doi.org/10.1007/s10641-013-0217-4}{10.1007/s10641-013-0217-4}.

\hypertarget{ref-Kronfeld}{}
Kronfeld-Schor, N., and Dayan, T. 2003. Partitioning of time as an
ecological resource. Annual Review of Ecology, Evolution, and
Systematics \textbf{34}: 153--181.
doi:\href{https://doi.org/10.1146/annurev.ecolsys.34.011802.132435}{10.1146/annurev.ecolsys.34.011802.132435}.

\hypertarget{ref-Larranaga2015}{}
Larranaga, N., and Steingrímsson, S.O. 2015. Shelter availability alters
diel activity and space use in a stream fish. Behavioral Ecology
\textbf{26}(2): 578--586.
doi:\href{https://doi.org/10.1093/beheco/aru234}{10.1093/beheco/aru234}.

\hypertarget{ref-McCubbins2016}{}
McCubbins, J.L., Hansen, M.J., DosSantos, J.M., and Dux, A.M. 2016.
Demographic characteristics of an adfluvial bull trout population in
Lake Pend Oreille, Idaho. North American Journal of Fisheries Management
\textbf{36}(6): 1269--1277.
doi:\href{https://doi.org/10.1080/02755947.2016.1209602}{10.1080/02755947.2016.1209602}.

\hypertarget{ref-McMillan2012}{}
McMillan, J.R., Dunham, J.B., Reeves, G.H., Mills, J.S., and Jordan,
C.E. 2012. Individual condition and stream temperature influence early
maturation of rainbow and steelhead trout, Oncorhynchus mykiss.
Environmental Biology of Fishes \textbf{93}(3): 343--355.
doi:\href{https://doi.org/10.1007/s10641-011-9921-0}{10.1007/s10641-011-9921-0}.

\hypertarget{ref-Mesa2013}{}
Mesa, M.G., Weiland, L.K., Christiansen, H.E., Sauter, S.T., and
Beauchamp, D.A. 2013. Development and Evaluation of a Bioenergetics
Model for Bull Trout. Transactions of the American Fisheries Society
\textbf{142}: 41--49.
doi:\href{https://doi.org/10.1080/00028487.2012.720628}{10.1080/00028487.2012.720628}.

\hypertarget{ref-Metcalfe1999}{}
Metcalfe, N.B., Fraser, N.H.C., and Burns, M.D. 1999. Food availability
and the nocturnal vs. diurnal foraging trade off in juvenile salmon.
Journal of Animal Ecology \textbf{68}: 371--381.

\hypertarget{ref-Monnet2020}{}
Monnet, G., Rosenfeld, J.S., and Richards, J.G. 2020. Adaptive
differentiation of growth, energetics and behaviour between piscivore
and insectivore juvenile rainbow trout (O. mykiss) along the
Pace‐of‐Life continuum. Journal of Animal Ecology \textbf{89}(11):
2717--2732.
doi:\href{https://doi.org/10.1111/1365-2656.13326}{10.1111/1365-2656.13326}.

\hypertarget{ref-Nakano1999a}{}
Nakano, S., Fausch, K.D., and Kitano, S. 1999. Flexible niche
partitioning via a foraging mode shift: a proposed mechanism of
coexistance in stream-dwelling charrs. Journal of Animal Ecology
\textbf{68}(6): 1079--1092.

\hypertarget{ref-Nakano1992}{}
Nakano, S., Fausch, K.D., and Tanaka, T. 1992. Resource utilization by
bull char and cutthroat trout in a mountain stream in Montana, USA.
Japanese Journal of Icthyology \textbf{39}(3): 211--217.

\hypertarget{ref-Naman2016}{}
Naman, S.M., Rosenfeld, J.S., and Richardson, J.S. 2016. Causes and
consequences of invertebrate drift in running waters: from individuals
to populations and trophic fluxes. Canadian Journal of Fisheries and
Aquatic Sciences \textbf{73}: 1292--1305.

\hypertarget{ref-Naman2019}{}
Naman, S.M., Rosenfeld, J.S., Neuswanger, J.R., Enders, E.C., and Eaton,
B.C. 2019a. Comparing correlative and bioenergetics‐based habitat
suitability models for drift‐feeding fishes. Freshwater Biology
\textbf{64}(9): 1613--1626.
doi:\href{https://doi.org/10.1111/fwb.13358}{10.1111/fwb.13358}.

\hypertarget{ref-Naman2020b}{}
Naman, S.M., Rosenfeld, J.S., Neuswanger, J.R., Enders, E.C., Hayes,
J.W., Goodwin, E.O., Jowett, I., and Eaton, B.C. 2020. Bioenergetic
habitat suitability curves for instream flow modelling: introducing
user‐friendly software and its potential applications. Fisheries: 1--9.
doi:\href{https://doi.org/10.1002/fsh.10489}{10.1002/fsh.10489}.

\hypertarget{ref-Naman2019b}{}
Naman, S.M., Ueda, R., and Sato, T. 2019b. Predation risk and resource
abundance mediate foraging behaviour and intraspecific resource
partitioning among consumers in dominance hierarchies. Oikos
\textbf{128}(7): 1005--1014.
doi:\href{https://doi.org/10.1111/oik.05954}{10.1111/oik.05954}.

\hypertarget{ref-Nestler2019}{}
Nestler, J.M., Milhous, R.T., Payne, T.R., and Smith, D.L. 2019. History
and review of the habitat suitability criteria curve in applied aquatic
ecology. River Research and Applications: 1--26.
doi:\href{https://doi.org/10.1002/rra.3509}{10.1002/rra.3509}.

\hypertarget{ref-Oskanen2013}{}
Oksanen, J., Kindt, R., Legendre, P., O'Hara, B., Simpson, G., Solymos,
P., Stevens, H., and Wagner, H. 2013. Vegan community ecology package. R
package version 2.0--9,
doi:\href{https://doi.org/http:/\%20/cran.r-project.org/package=vegan}{http:\textbackslash{} \textbackslash{}cran.r-project.org\textbackslash{}package=vegan}.

\hypertarget{ref-Piccolo2014}{}
Piccolo, J.J., Frank, B.M., and Hayes, J.W. 2014. Food and space
revisited: the role of drift-feeding theory in predicting the
distribution, growth, and abundance of stream salmonids. Environmental
Biology of Fishes \textbf{97}(5): 475--488.
doi:\href{https://doi.org/10.1007/s10641-014-0222-2}{10.1007/s10641-014-0222-2}.

\hypertarget{ref-Pinto2013}{}
Pinto, M.C., Post, J.R., Paul, A.J., Johnston, F.D., Mushens, C.J., and
Stelfox, J.D. 2013. Lateral and Longitudinal Displacement of
Stream-Rearing Juvenile Bull Trout in Response to Upstream Migration of
Spawning Adults. Transactions of the American Fisheries Society
\textbf{142}(6): 1590--1601.
doi:\href{https://doi.org/10.1080/00028487.2013.822419}{10.1080/00028487.2013.822419}.

\hypertarget{ref-Polacek2003a}{}
Polacek, M.C., and James, P.W. 2003. Diel microhabitat use of age-O bull
trout in Indian Creek, Washington. Ecology of Freshwater Fish
\textbf{12}(1): 81--86.
doi:\href{https://doi.org/10.1034/j.1600-0633.2003.00004.x}{10.1034/j.1600-0633.2003.00004.x}.

\hypertarget{ref-Quinn2018}{}
Quinn, T.P. 2018. Behavior and Ecology of Pacific Salmon and Trout.
\emph{In} 2nd editions. University of Washington Press.

\hypertarget{ref-Railsback2020}{}
Railsback, S.F., Harvey, B.C., and Ayllón, D. 2020a. Contingent
trade-off decisions with feedbacks in cyclical environments: testing
alternative theories. Behavioral Ecology: 1--15.
doi:\href{https://doi.org/10.1093/beheco/araa070}{10.1093/beheco/araa070}.

\hypertarget{ref-Railsback2020b}{}
Railsback, S.F., Harvey, B.C., and Ayllón, D. 2020b. Importance of the
Daily Light Cycle in Population‐Habitat Relations: A Simulation Study.
Transactions of the American Fisheries Society.
doi:\href{https://doi.org/10.1002/tafs.10283}{10.1002/tafs.10283}.

\hypertarget{ref-Railsback2005}{}
Railsback, S.F., Harvey, B.C., Hayse, J.W., and Lagory, K.E. 2005. Tests
of theory for diel variation in salmonid feeding activity and habitat
use. Ecology \textbf{86}(4): 947--959.

\hypertarget{ref-Reale2010}{}
Réale, D., Garant, D., Humphries, M.M., Bergeron, P., Careau, V., and
Montiglio, P.O. 2010. Personality and the emergence of the pace-of-life
syndrome concept at the population level. Philosophical Transactions of
the Royal Society B: Biological Sciences \textbf{365}(1560): 4051--4063.
doi:\href{https://doi.org/10.1098/rstb.2010.0208}{10.1098/rstb.2010.0208}.

\hypertarget{ref-Rieman1993a}{}
Rieman, B.E., and McIntyre, J.D. 1993. Demographic and habitat
requirements for conservation of bull trout. General Technical Report -
US Department of Agriculture, Forest Service (INT-302).

\hypertarget{ref-Rosenfeld2020}{}
Rosenfeld, J.S., Richards, J.G., Allen, D., Van Leeuwen, T.E., and
Monnet, G. 2020. Adaptive trade-offs in fish energetics and physiology:
insights from adaptive differentiation among juvenile salmonids.
Canadian Journal of Fisheries and Aquatic Sciences \textbf{2}: 1--47.
doi:\href{https://doi.org/10.1139/cjfas-2019-0350}{10.1139/cjfas-2019-0350}.

\hypertarget{ref-Roussel1999}{}
Roussel, J.M., Bardonnet, A., and Claude, A. 1999. Microhabitats of
brown trout when feeding on drift and when resting in a lowland salmonid
brook: effects on Weighted Usable Area. Archiv für Hydrobiologie
\textbf{146}(4): 413--429.

\hypertarget{ref-Sato2014}{}
Sato, T., and Watanabe, K. 2014. Do stage-specific functional responses
of consumers dampen the effects of subsidies on trophic cascades in
streams? Journal of Animal Ecology \textbf{83}(4): 907--915.
doi:\href{https://doi.org/10.1111/1365-2656.12192}{10.1111/1365-2656.12192}.

\hypertarget{ref-Tyler2008}{}
Tyler, J.A., and Bolduc, M.B. 2008. Individual Variation in Bioenergetic
Rates of Young-of-Year Rainbow Trout. Transactions of the American
Fisheries Society \textbf{137}(1): 314--323.
doi:\href{https://doi.org/10.1577/t05-238.1}{10.1577/t05-238.1}.

\hypertarget{ref-Walters1993a}{}
Walters, C.J., and Juanes, F. 1993. Recruitment limitation as a
consequence of natural selection for use of restricted feeding habitats
and predation risk taking by juvenile fishes. Canadian Journal of
Fisheries and Aquatic Sciences \textbf{50}: 2058--2070.

\hypertarget{ref-Wilzbach1986a}{}
Wilzbach, M., Cummins, K., and Hall, J. 1986. Influence of habitat
manipulations on interactions between cutthroat trout and invertebrate
drift. Ecology \textbf{67}(4): 898--911.

\hypertarget{ref-Wilzbach1985}{}
Wilzbach, M.A. 1985. Relative roles of food abundance and cover in
determining the habitat distribution of stream-dwelling cutthroat trout
(Salmo clarki). Canadian Journal of Fisheries and Aquatic Sciences
\textbf{42}: 1668--1672.

\hypertarget{ref-Young2004}{}
Young, K.A. 2004. Asymmetric competition, habitat selection, and niche
overlap in juvenile salmonids. Ecology \textbf{85}(1): 134--149.

\section{Figures}\label{figures}

\textbf{Figure 1} Photographs of experimental cover boxes. Branches were
attached to the underside of each box to simulate cover from wood (left
panel). Boxes were anchored to the substrate with rebar, with the open
sides of the box facing upstream perpendicular to flow.

\textbf{Figure 2} Activity (top panel) and cover use (bottom panel) as
proportions of the total number of individual bull trout and rainbow
trout observed in each time period.

\textbf{Figure 3} Boxplots showing focal velocities (top panel) and
total water depth (bottom panel) occupied by foraging and resting
rainbow trout in each time period. Points represent an individual fish
and are jittered slightly to ease interpretation. Boxes represent the 25
and 75\% quartiles and lines inside the boxes represent the medians.

\textbf{Figure 4} Constrained Analysis of Principal Coordinates (CAP)
plots for resting and foraging bull trout and rainbow trout in each time
period. Arrows represent the relative contributions of different habitat
factors to the constrained ordination axes, points represent the
positions of individual fish observations in the ordination space, and
ellipses represent the bivariate standard deviation for each of the four
groups representing a combination of species and activity. Separate
ordinations were performed in each time period, thus the CAP axes are on
different scales for each plot.

\textbf{Figure 5} The probability that an experimentally added cover
unit will be colonized by a foraging rainbow trout as a function of the
estimated net rate of energy intake (J s\(^{-1}\)) at the location the
cover unit was placed. Points indicate individual cover boxes. The trend
line and shaded region of 95\% confidence are predicted from the
binomial GLM.

\textbf{Figure 6} Univariate habitat suitability curves (HSCs) for
depth, velocity, substrate, and cover for rainbow trout and bull trout
compared across the three time periods (day, dusk, night). HSCs are
based on GLM modele predictions of observed habitat use relative to
availability and are standardized to a maximum of 1.

\end{document}
